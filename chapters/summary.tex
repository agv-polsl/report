W ramach projektu zaprojektowano model 3D robota mobilnego, a następnie
stworzono symulację, która pozwala na implementację oraz testowanie
poszczególnych algorytmów.

Wykorzystując platformę ROS stworzono architekturę węzłów, która pozwoli na
łatwą integrację z rzeczywistym obiektem, co może być podstawą do kontynuowania
projektu.

Za pomocą wyznaczonego modelu kinematycznego pojazdu stworzono węzeł, który
określa odometrie pojazdu, na podstawie symulowanych danych z enkodera.

Opracowano system AHRS bazujący na filtrze Kalmana.
System wyznaczania orientacji w~przestrzeni zaimplementowano i~przetestowano
z~użyciem systemu mikroprocesorowego.

Wykonano prototyp systemu wizyjnego wyznaczającego przestrzeń dozwoloną robota
w~labiryncie.

W~raporcie załączono wnioski wynikające z~przeprowadzonej analizy rynku
pojazdów autonomicznych oraz analizę ryzyka przedsięwzięcia tego typu.
