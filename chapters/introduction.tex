Przedmiotem projektu jest konstrukcja zestawu wybranych algorytmów robotyki dla
mobilnych pojazdów autonomicznych oraz ich symulacja w środowisku ROS (Robotics
Operating
System). Platforma ROS pozwala na szybkie prototypowanie oprogramowania robotów
w wirtualnych środowiskach symulacyjnych z~wykorzystaniem systemu operacyjnego
Linux.
Modelowanie obiektów i~konstrukcja algorytmów w językach programowania Python
oraz C++ z użyciem systemu ROS daje możliwość łatwego przenoszenia
oprogramowania na systemy wbudowane.

Wybrane algorytmy obejmują tematy predykcji położenia, przetwarzania danych
wizyjnych, wykrywania elementów otoczenia oraz planowania toru jazdy. Wyniki
projektu mogą być podstawą do konstrukcji platformy sprzętowej realizującej
algorytmy. Gotowa platforma może stanowić podstawę do dalszych prac nad
wykorzystaniem sztucznej inteligencji i~algorytmów sterowania w~systemach
mobilnych.

Dodatkowo, w~ramach projektu, zrealizowano analizę biznesową produkcji robotów
autonomicznych w~lokalnych warunkach, na podstawie danych o istniejących
firmach działających w~rozpatrywanej dziedzinie.

Środowisko ROS znajduje zastosowania w przemyśle oraz jest narzędziem
wykorzystywanym przy pracach nad dynamicznie rozwijającą się technologią
pojazdów autonomicznych.

\chapter{Organizacja}


\section{Podział zadań}
Podział zadań między studentów przedstawiono w~tabeli \ref{tab:org}.
Ilość zadań przydzielonych zadań oraz ich charakter odpowiada efektom
kształcenia realizowanym przez studentów.

\begin{table}[h]
\begin{tabular}{ll}
\toprule
\multicolumn{1}{c}{Wojciech Ptasiński}   & \multicolumn{1}{c}{Maciej Ziaja} \\
\midrule
Konfiguracja środowiska ROS              & Przygotowanie skryptów
                                           konfiguracyjnych \\
Przygotowanie modelu robota i symulatora & Implementacja sterowników czujników
                                           IMU \\
Opracowanie odometrii robota             & Implementacja filtru Kalmana \\
Implementacja sterowania robotem         & Opracowanie i implementacja systemu
                                           AHRS \\
Analiza biznesowa                        & Opracowanie prototypu systemu
                                           wizyjnego \\
                                         & Przygotowanie prototypu sprzętowego
                                           mikroprocesora \\
                                         & Analiza biznesowa \\
\bottomrule
\end{tabular}
\caption{Tabela podziału zadań między studentów realizujących projekt}
\label{tab:org}
\end{table}