Przedmiotem projektu jest konstrukcja zestawu wybranych algorytmów robotyki dla
mobilnych pojazdów autonomicznych oraz ich symulacja w środowisku ROS (Robotics
Operating
System). Platforma ROS pozwala na szybkie prototypowanie oprogramowania robotów
w wirtualnych środowiskach symulacyjnych z~wykorzystaniem systemu operacyjnego
Linux.
Modelowanie obiektów i~konstrukcja algorytmów w językach programowania Python
oraz C++ z użyciem systemu ROS daje możliwość łatwego przenoszenia
oprogramowania na systemy wbudowane.

Wybrane algorytmy obejmują tematy predykcji położenia, przetwarzania danych
wizyjnych, wykrywania elementów otoczenia oraz planowania toru jazdy. Wyniki
projektu mogą być podstawą do konstrukcji platformy sprzętowej realizującej
algorytmy. Gotowa platforma może stanowić podstawę do dalszych prac nad
wykorzystaniem sztucznej inteligencji i~algorytmów sterowania w systemach
mobilnych.

Dodatkowo, w~ramach projektu, zrealizowano analizę biznesową produkcji robotów
autonomicznych w~lokalnych warunkach, na podstawie danych o istniejących
firmach działających w rozpatrywanej dziedzinie.

Środowisko ROS znajduje zastosowania w przemyśle oraz jest narzędziem
wykorzystywanym
przy pracach nad dynamicznie rozwijającą się technologią pojazdów
autonomicznych.
