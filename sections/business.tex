W~ramach projektu dokonano analizy możliwości biznesowych związanych z~robotyką
mobilną w~regionie Górnego Śląska.
Pełną treść analizy zamieszczono w~osobnym dokumencie.

\subsubsection{Przegląd dostępnych źródeł finansowania}
Przeanalizowano ofertę dofinansowania projektów oraz przedsięwzięć biznesowych:
zarówno krajowych, jak i zagranicznych.
W~ramach przeglądu wzięto pod uwagę oferty aktualne oraz niedawno zakończone.
Dofinansowania w~ramach programów rozwoju oraz inkubatorów charakteryzowały się
atrakcyjnymi ofertami, niektóre z~nich nie wymagały wkładu własnego, inne
uzależniały go od wysokości kwoty pomocy.
Wkład własny większości rozpatrywanych programów nie przekraczał parunastu
procent.
Oferty proponowały wsparcie w~wysokości od parudziesięciu do paruset tysięcy
złotych.
Częstym wymogiem uczestnictwa w~programie była gotowości prototypu produktu.
Wsparcie w~projektach jest ograniczone do małych firm, przy czym pożądana jest
współpraca z~uczelnią.
W~ramach przeglądu analizowano programy takie jak: \textit{POWER},
\textit{Program Akceleracyjny KPT} oraz \textit{Program Inteligentny Rozwój}.

\subsubsection{Przykłady rozwijających się firm na rynku robotyki}
Spośród polskich działalności przeanalizowano rozwój firm takich jak
\textit{KP Labs} oraz \textit{Future Processing}.
Zwrócono uwagę na źródła finansowania projektów.
Produkty o~wysokiej innowacyjności i~aspekcie naukowym (takie jak w~przypadku
firmy \textit{KP Labs}) uzyskały bardzo wysokie dofinansowania unijne,
przekraczające połowę wartości poszczególnych przedsięwzięć.
Produkty związane na przykład z~kosmonautyką mogły liczyć na dotacje wielkości
milionów złotych.

Zdecydowano się także na analizę zagranicznych firm, które w~ostatnich latach
odniosły sukces na arenie międzynarodowej.
Zwrócono uwagę na działalność skandynawskiej firmy \textit{Antmicro}, która
posiada polskie filie.
Wiele z~wykorzystywanych przez firmę technologii należy do grupy \textit{open
source} oraz \textit{open source hardware}.
Ciekawym przypadkiem okazała się inicjatywa \textit{CommaAI}, której twórcą
jest znany programista Georg Hotz.
Jego firma skupia się na wyposażaniu istniejących samochodów, w~systemy jazdy
autonomicznej.
\textit{CommaAI} oferuje oprogramowanie oraz moduły współpracujące z~czujnikami
i~kamerami wbudowanymi w~samochody oraz smartfony.
Jest to ciekawe podejście wykorzystujące wykosi poziom cyfryzacji otaczających
technologii.
Takie podejście ułatwia tworzenie produktów oraz pozwala na oferowanie ich
w~konkurencyjnych cenach.

\subsubsection{Analiza SWOT potencjalnego przedsięwzięcia}
W~ramach analizy przeprowadzono analizę \textit{SWOT} hipotetycznego
przedsięwzięcia w~dziedzinie robotyki mobilnej w~warunkach biznesowych Górnego
Śląska.
Analiza skupia się na możliwościach założenia działalności biznesowej przez
absolwentów Politechniki Śląskiej.

\paragraph{Mocne strony}
\begin{itemize}
	\item potencjalni członkowie zespołu, tzn. absolwenci kierunków technicznych
		  są zapoznani z~bieżącymi, nowoczesnymi technologiami,
	\item potencjalni członkowie zespołu rozumieją potrzeby nowoczesnego
		  przemysłu.
	\item dobry wewnętrzny kontakt absolwentów z uczelnią.
\end{itemize}

\paragraph{Słabe strony}
\begin{itemize}
	\item potencjalni członkowie zespołu (absolwenci, doktoranci) mają małe
          doświadczenie biznesowe i~menadżerskie.
\end{itemize}

\paragraph{Szanse}
\begin{itemize}
	\item zainteresowanie medialne i społeczne nowymi technologiami,
	\item liczne projekty dofinansowania; unijne oraz krajowe,
    \item możliwość współpracy z jednostkami naukowymi i uczelniami badawczymi.
\end{itemize}

\paragraph{Zagrożenia}
\begin{itemize}
	\item nieprzygotowanie polskiego rynku na adaptowanie nowych technologii,
	\item prawdopodobny kryzys,
	\item ryzyko przejęcia firmy przez zagraniczny koncern, ze względu na brak
          bezpośredniego odbiorcy zaawansowanych produktów na rodzimym rynku.
\end{itemize}