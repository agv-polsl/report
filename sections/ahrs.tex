System AHRS (ang. \textit{attitude and heading reference system}) pozwala na
wyznaczenie orientacji obiektu w~przestrzeni.
Systemu tego typu znajdują szerokie zastosowanie w~pojazdach autonomicznych.

System AHRS można podzielić na dwie główne części: system czujników oraz system
predykcji orientacji w~przestrzeni.
Fragment systemu odpowiedzialny za działanie czujników stawia wyzwania związane
z~obsługą sprzętu oraz jego kalibracją.
Algorytm predykcji pozwala na fuzję danych z~zespołu czujników w~celu
określenia orientacji obiektu w~przestrzeni.

Wynikiem działania systemu AHRS jest wektor trzech wartości reprezentujących
obrót obiektu wokół osi trójwymiarowego układu współrzędnych.
Kąty obrotu noszą nazwę \textit{kątów Eulera}, ich ideowym odpowiednikiem
w~terminologii lotniczej są kąty \textit{roll, pitch, yaw}.

\subsection{Czujniki systemu AHRS i implementacja sprzętowa}
W~celu określenia orientacji obiektu w~przestrzeni konieczny jest zestaw
czujników.
Do wyznaczenia pełnej orientacji w~przestrzeni trójwymiarowej stosuje się
zazwyczaj zestaw: akcelerometru, żyroskopu oraz magnetometru.
Akcelerometr i~żyroskop tworzą razem zestaw nazywany IMU (ang \textit{inertial
measurement unit}).
System IMU pozwala na wyznaczenie kątów odchylenia obiektu od pionu
(\textit{roll} oraz \textit{pitch}), co w~wielu zastosowaniach jest
wystarczające.
Aby uzyskać pełne współrzędne orientacji obiektu w~przestrzeni należy użyć
także magnetometru, który pozwala określić kąt obrotu wokół osi pionowej
(\textit{yaw}).

W~projekcie użyto układu \textit{MinIMU-9 v5} firmy \textit{Pololu}
o~dziewięciu stopniach swobody (trzy stopnie swobody na każdy z~czujników).
\textit{MinIMU} zawiera zestaw: układ IMU \textit{LSM6DS33} oraz magnetometr
\textit{LIS3MDL}.
Komunikacja z czujnikami odbywa się poprzez magistralę \textit{I2C}, przy czym
każdy z trzech czujników ma osobny adres i~wymaga oddzielnej obsługi
komunikacji.
Czujniki wchodzące w skład systemu AHRS wymagają kalibracji, co jest
szczególnie ważne przypadku magnetometru.

\subsubsection{Żyroskop}
Żyroskop to przyrząd pozwalający określić prędkość obiektu kątową w~trzech
osiach.
W~spoczynku żyroskop powinien dawać zerowe odczyty ze wszystkich osi.
Jest to podstawowy czujnik pozwalający określić orientację obiektu
w~przestrzeni.
Wyznaczenie kątów obrotu wymaga jedynie całkowania odczytu z~żyroskopu
(ponieważ prędkość kątowa to pochodna kąta obrotu).
Niestety sam żyroskop nie jest wystarczający aby precyzyjnie orientować obiekt
w~przestrzeni, ze względu na zjawisko dryftu (ang. \textit{gyroscope drift}).
Żyroskop jest mało podatny na szumy pomiarowe o~małej częstotliwości, natomiast
jest wrażliwy na pojawianie się stałych, ale niewielkich błędów w~odczytach.
Ze względu na konieczność całkowania danych z~żyroskopu te małe błędy są
kumulowane, w~wyniku czego wraz z~upływem czasu, orientacja na podstawie
działania żyroskopu staje się coraz mniej wiarygodna.
Dlatego w celu precyzyjnego ustalania orientacji w~przestrzeni konieczne jest
użycie żyroskopu w~parze z akcelerometrem oraz zastosowanie algorytmu fuzji
danych.

\subsubsection{Akcelerometr}
Akcelerometr to czujnik pozwalający mierzyć przyspieszenie kątowe obiektu.
W stanie spoczynku osie poziomie akcelerometru powinny wskazywać zerowe
odczyty, natomiast oś pionowa powinna wskazywać wartość przyspieszenia
ziemskiego.

Akcelerometr jest wolny od zjawiska dryftu, natomiast jest wrażliwy na szumy
pomiarowe o~dużych częstotliwościach.
Przeciwna natura niedokładności akcelerometru oraz żyroskopu sprawia, że
najlepiej pracują one w parze, z~użyciem algorytmu fuzji danych.

W~czujnikach IMU często dokonuje się kalibracji błędu zera (addytywnego).
Należy wykonać to w trakcie spoczynku systemu na płaskiej powierzchni.
Oś pionowa akcelerometru nie powinna być kalibrowana.

\subsubsection{Magnetometr}
Magnetometr mierzy indukcję pola magnetycznego w~trzech kierunkach.
Ponieważ między biegunami Ziemi przebiegają linie pola magnetycznego, to
magnetometr może być użyty do wyznaczenia orientacji obiektu w przestrzeni
(jako kompas).
Oś magnetometru leżąca w kierunku północ-południe wskazuje wartości maksymalne
indukcji, osie prostopadłe wskazują wartości minimalne.

Kluczowym zagadnieniem jest kalibracja magnetometru, który jest obciążony
zarówno błędem addytywnym (nazywanym z ang. \textit{hard iron}), jak
i~multiplikatywnym (nazywanym z ang. \textit{soft iron}).
Odczyty magnetometru są wrażliwe na zakłócenia powstałe w wyniku działań innych
części układów robotyki.
Z~tego powodu kalibracja magnetometru jest koniecznością, magnetometr
nieskalibrowany jest obciążony błędami uniemożliwiającymi poprawne działanie
systemu AHRS.

\subsection{Kalibracja czujników}
W przypadku czujników IMU kalibracja polega na uśrednieniu odczytu $ n $
wartości w~trakcie spoczynku na równym podłożu.
Kolejne odczyty należy korygować o~zapisaną wartość średniej.

W przypadku magnetometru w czasie kalibracji należy nim poruszać, tak aby
każda oś jego pomiaru obróciła się w~przestrzeni o~kąt pełny.
W~przypadku poprawnie skalibrowanego magnetometru pomiary o~maksymalnych
wartościach powinny mieć na wykresie kształt okręgu ze środkiem w~początku
układu współrzędnych.
W przypadku wystąpienia błędu \textit{hard iron} środek okręgu jest
przesunięty, natomiast w~wyniku działania błędu \textit{soft iron} następuje
zniekształcenie okręgu.

Ważne jest aby kalibracji dokonać w~odpowiedniej kolejności, najpierw należy
wyznaczyć \textit{hard iron}, a~potem \textit{soft iron}.
Obliczenie błędu multiplikatywnego wymaga, aby błąd addytywny był już
zniwelowany.
Błąd \textit{hard iron} jest średnią ze skrajnych wartości $ n $ pomiarów, co
przedstawiono na wzorze \ref{eq:hard_iron}, gdzie $ \vec{m_n} $ to wektor
pomiarów z~czujnika.
\begin{equation}
	\vec{e_{hard}} = 
		\frac{max\left(\vec{m_n}\right) + min\left(\vec{m_n}\right)}{2}
\label{eq:hard_iron}
\end{equation}
Błąd \textit{soft iron} należy wyznaczyć według wzorów od
\ref{eq:hard_iron_beg} do \ref{eq:hard_iron_end}.
\begin{gather}
	\vec{r} =
		\frac{max\left(\vec{m_n}\right) - min\left(\vec{m_n}\right)}{2}
	\quad
	\vec{r} = \left[r_x, r_y, r_z\right] \\
\label{eq:hard_iron_beg}
	r_a = \frac{r_x + r_y + r_z}{3} \\
	\vec{e_{soft}} =
		\left[ \frac{r_a}{r_x}, \frac{r_a}{r_y}, \frac{r_a}{r_z}\right]
\label{eq:hard_iron_end}
\end{gather}
Ostatecznie odczyty magnetometru należy korygować według wzoru
\begin{equation}
	\vec{m_{fixed}} = \left(\vec{m} - \vec{e_{hard}}\right) \cdot \vec{e_{soft}}
\end{equation}

\subsection{Implementacja obsługi czujników w systemie operacyjnym Linux}
Obsługę czujników zdecydowano się oprzeć na systemie Linux, który jest również
bazą dla środowiska ROS.
Jądro systemu Linux udostępnia dostęp do magistrali komunikacji \textit{I2C}
poprzez wywołania systemowe (ang. \textit{system calls}).
Urządzenia peryferyjne w~systemach pochodnych systemu \textit{UNIX} są
reprezentowane przez pliki znakowe.
W~celu nawiązania połączenia z~czujnikiem należy otworzyć plik urządzenia
znakowego \textit{I2C} za pomocą wywołania \mintinline{c}{open()}.
Pliki urządzeń peryferyjnych znajdują się w~katalogu \texttt{/dev/}.
Połączenie nawiązuje się wywołaniem
\mintinline{C}{ioctl(device_handle, I2C_SLAVE, i2c_address);}, gdzie
\mintinline{C}{device_handle} to deskryptor otwartego pliku urządzenia
znakowego, \mintinline{C}{I2C_SLAVE} to stała określająca tryb działania
urządzania, a \mintinline{C}{i2c_address} to adres urządzenia podłączonego do
magistrali \textit{I2C}.
Wymiana z~urządzeniami odbywa się poprzez zapis i~odczyt danych z pliku
urządzenia, za pomocą wywołań \mintinline{C}{read()} oraz
\mintinline{C}{write}.
Istotne jest aby odczytywać i~zapisywać odpowiednią liczbę bajtów, oraz
poprawnie skalować odczyty.
Informacje na temat formatu przesyłanych danych znajdują się w~notach
katalogowych czujników.
Istnieją dodatkowe biblioteki ułatwiające pracę z~urządzeniami \textit{I2C}
(np. \textit{SMBUS}), jednak opisany sposób jest najbardziej uniwersalny.
Utworzona biblioteka komunikacji z~czujnikami \textit{MinIMU-9 v5} będzie
działała na każdym urządzeniu z~systemem Linux, którego jądro obsługuje
komunikację \textit{I2C}.

Program napisano w~języku C\texttt{++} odpowiednio opakowując kod języka C.
Błędy zgłaszane przez system \textit{errno} obsłużono przez wyjątki.
Adresy rejestrów i~urządzeń \textit{I2C} zapisano w~silnie typowanych
wyrażeniach wyliczeniowych, które parametryzują szablony klas urządzeń
magistrali \textit{I2C}.
Oprogramowanie obsługi czujników udostępniono w~postaci statycznie łączonej
biblioteki programistycznej.

\subsection{Filtr Kalmana}
Filtr Kalma jest algorytmem estymacji stanu modelu dynamicznego.
Algorytmy tego typu określają wewnętrzny stan obiektu na podstawie pomiarów
wejścia i~wyjścia tego układu.
Filtr wykonuje estymacje rekurencyjnie, na podstawie wcześniejszych obliczeń,
a~szacunki są optymalne, zakładając że błędy pomiarowe mają rozkład
normalny oraz układ jest liniowy.
Jeżeli układ dyskretny opisany jest równaniami
\ref{eq:state_kalman}~i~\ref{eq:meas_kalman}, gdzie:
\begin{description}
	\item[$x\left(t\right)$] to wektor stanu w~funkcji czasu,
	\item[$y\left(t\right)$] to wektor wyjścia układu (pomiaru) w~funkcji czasu,
	\item[$A$] to macierz systemowa (przejścia),
	\item[$B$] to macierz wejścia,
	\item[$H$] to macierz wyjścia (pomiaru),
	\item[$v\left(t\right), u\left(t\right)$] to wektory szumu, kolejno
		procesu oraz pomiaru.	
\end{description}
\begin{align}
x\left(t + 1\right) &= Ax\left(t\right) + Bu\left(t\right) + v\left(t\right)
\label{eq:state_kalman} \\
y\left(t\right) &= Hx\left(t\right) + w\left(t\right)
\label{eq:meas_kalman}
\end{align}

Działanie filtru Kalmana składa się z~dwóch faz: fazy predykcji (\textit{a
priori}) oraz aktualizacji (\textit{a posteriori}).
Faza predykcji (ekstrapolacji) polega na wyznaczeniu stanu układu w~nowej
chwili czasu na podstawie modelu układu oraz wcześniejszego stanu.
W~czasie predykcji wyznaczana jest również nowa macierz niepewności
(kowariancji estymacji).
Predykcja odbywa się według wzorów
\ref{eq:predict_state}~i~\ref{eq:predict_unc}, gdzie:
\begin{description}
	\item[$\hat{x}_a\left(t\right)$] to estymacja \textit{a priori} wektora stanu
	w~funkcji czasu,
	\item[$\hat{x}_p\left(t\right)$] to estymacja \textit{a posteriori} wektora
	stanu w~funkcji czasu,
	\item[$P_a\left(t\right)$] to macierz kowariancji estymacji stanu \textit{a
	priori} w~funkcji czasu,
	\item[$Q$] to macierz kowariancji szumu procesu.
\end{description}
\begin{align}
	\hat{x}_a\left(t\right) &= A\hat{x}_p\left(t-1\right) + Bu\left(t-1\right)
	\label{eq:predict_state} \\
	P_a\left(t\right) &= AP_p\left(t-1\right)A^T + Q
	\label{eq:predict_unc}
\end{align}
Kolejnym etapem jest aktualizacja, która odbywa się na podstawie pomiaru
wyjścia układu.
Ma ona za zadanie poprawić estymację stanu, uwzględniając różnicę między
rzeczywistym pomiarem wyjścia, a~jego wartością obliczoną \textit{a priori}
przy pomocy stanu z~fazy predykcji.
Kluczowym elementem filtru Kalmana jest \textit{wzmocnienie Kalmana}, które
reprezentuje balans ufności w~predykcję \textit{a priori} (według modelu) oraz
w~pomiary wyjścia \textit{a posteriori}.
W~trakcie fazy aktualizacji, wyznaczane jest także nowe
\textit{wzmocnienie Kalmana}.
Intuicyjnie można powiedzieć, że wzmocnienie jest wyznaczane tak, by najlepiej
balansować pomiędzy przewidywaniem stanu według modelu matematycznego
(niedoskonałego, a~także nie obejmującego zakłóceń procesu) oraz przewidywaniem
według pomiaru (obarczonego szumem pomiarowym).
Faza aktualizacji odbywa się zgodnie ze wzorami od \ref{eq:kalman_update_K} do
\ref{eq:kalman_update_P}, gdzie:
\begin{description}
	\item[$K\left(t\right)$] to \textit{wzmocnienie Kalmana} w~funkcji czasu,
	\item[$\hat{P}_p\left(t\right)$] to macierz kowariancji estymacji stanu
	\textit{a posteriori} w~funkcji czasu,
	\item[$R$] to macierz kowariancji szumu pomiarowego.
\end{description}
\begin{gather}
	K\left(t\right) = P\left(t\right) H^T \cdot
	\left(HP_a\left(t\right)H^T + R\right)^{-1}
\label{eq:kalman_update_K} \\
	\hat{x}_p\left(t\right) = \hat{x}_a\left(t\right) +
	K\cdot\left(y\left(t\right) - H \hat{x}_a\left(t\right)\right)
\label{eq:kalman_update_x} \\
	P_p\left(t\right) = \left(I - K\left(t\right)H\right) \cdot P_a\left(t\right)
	\label{eq:kalman_update_P}
\end{gather}

\subsection{Zastosowanie filtru Kalmana w~fuzji danych pochodzących z~czujnika
            IMU}
Filtr Kalmana można zastosować, aby oszacować orientację obiektu w~przestrzeni
na podstawie odczytów z~czujnika IMU.
Fuzja danych przy pomocy filtru pozwala zniwelować niedoskonałości żyroskopu
(dryft) oraz akcelerometru (szumy pomiarowe).
Zazwyczaj dane z~żyroskopu są używane w roli wejścia systemu, na etapie
predykcji, a~pomiary z~akcelerometru są wykorzystywane podczas fazy
aktualizacji.
Czujniki IMU pozwalają jedynie na określenie kątów \textit{roll} oraz
\textit{pitch}, w~dalszej części raportu zostanie opisany sposób wyznaczenia
kąta \textit{yaw} na podstawie odczytów magnetometru.

\subsubsection{Dane wejściowe filtru Kalmana}
Na podstawie danych z~akcelerometru można obliczyć dwa z~kątów opisujących
orientację obiektu w~przestrzeni (wykorzystując fakt, stałego przyspieszenia
ziemskiego w~pionowej osi inercyjnego układu odniesienia).
Kąty \textit{roll} i~\textit{pitch} są wyznaczane według wzorów
\ref{eq:acc_roll}~oraz \ref{eq:acc_pitch}, gdzie gdzie: $ \phi $ i~$ \theta $
to kolejno kąty \textit{roll oraz pitch}, a~$ a $~to odczyty akcelerometru.
\begin{align}
	\tilde\phi_a &= arctan\left(\frac{a_y}{\sqrt{a_x^2 + a_z^2}}\right)
	\cdot
	\frac{180}{\pi}
	\label{eq:acc_roll} \\
	\tilde\theta_a &= arctan\left(\frac{a_x}{\sqrt{a_y^2 + a_z^2}}\right)
	\cdot
	\frac{180}{\pi}
	\label{eq:acc_pitch}
\end{align}

Prędkości kątowe odczytane z~żyroskopu nie mogą być użyte wprost, ze względu
na powiązanie osi odczytu w~układzie odniesienia.
Obrót układu wokół jednej z~jego osi może powodować zmianę orientacji obiektu
w~wielu osiach zewnętrznego układu odniesienia.
Transformacji prędkości kątowych z~układu żyroskopu do układu inercyjnego
należy dokonać według wzoru \ref{eq:gyro_tf}, gdzie: $ \phi, \theta, \psi $ to
kolejno kąty \textit{roll, pitch oraz yaw}, a~$ g $~to odczyty żyroskopu.
\begin{equation}
	\begin{bmatrix}
		\dot{\phi_g} \\
		\dot{\theta_g} \\
		\dot{\psi_g}
	\end{bmatrix}
	=
	\begin{bmatrix}
		1 & sin\phi tan\theta & cos\phi tan\theta \\
		0 & cos\phi           & -sin\phi \\
		0 & sin\phi sec\theta & cos\phi sec\theta
	\end{bmatrix}
	\cdot
	\begin{bmatrix}
		g_x \\
		g_y \\
		g_z
	\end{bmatrix}
\label{eq:gyro_tf}
\end{equation}

\subsubsection{Równania filtru Kalmana dla fuzji danych}
Na podstawie opisu teoretycznego filtru Kalmana oraz przygotowanych danych
z~czujników można przygotować ostateczną wersję systemu podstawiając konkretne
macierze systemowe i~wektory z~równań: \ref{eq:state_imu} i~\ref{eq:vec_imu} do
równań od \ref{eq:state_kalman} do \ref{eq:kalman_update_P}.
Symbol $ \Delta t $ oznacza okres próbkowania czujników.
\begin{gather}
	A = 
	\begin{bmatrix}
		1 & -\Delta t & 0 & 0 \\
		0 & 1         & 0 & 0 \\
		0 & 0          & 1 & -\Delta t \\
		0 & 0         & 0 & 1 
	\end{bmatrix}
	\quad
	B = 
	\begin{bmatrix}
		\Delta t & 0 \\
		0        & 0 \\
		0        & \Delta t \\
		0        & 0
	\end{bmatrix}
	\quad
	H =
	\begin{bmatrix}
		1 & 0 & 0 & 0 \\
		0 & 0 & 1 & 0
	\end{bmatrix}
\label{eq:state_imu} \\
	\vec{x}\left(t\right) = 
	\begin{bmatrix}
		\hat{\phi}\left(t\right) \\
		b_{\hat{\phi}\left(t\right)} \\
		\hat{\theta}\left(t\right) \\
		b_{\hat{\theta}\left(t\right)}
	\end{bmatrix}
	\quad
	\vec{u}\left(t\right) = 
	\begin{bmatrix}
		\dot{\phi_{g}}\left(t\right) \\
		\dot{\theta_{g}}\left(t\right)
	\end{bmatrix}
	\quad
	\vec{y_t}
	\begin{bmatrix}
		\tilde{\phi_{a}}\left(t\right) \\
		\tilde{\theta_{a}}\left(t\right)
	\end{bmatrix}
\label{eq:vec_imu}
\end{gather}

Macierze $ Q $ (kowariancja szumu procesu) oraz $ R $ (kowariancja szumu
pomiaru) są dobierane przez użytkownika.
Są to zazwyczaj macierze diagonalne, im większa wartość na ich przekątnej, tym
większe zakłócenia w~danej części układu.
Początkowa wartość na przekątnej $ P $ reprezentuje wstępną ufność
w~dokładność modelu.

\subsection{Algorytm kompasu oraz kompensacja przechylenia}

Zgodnie z~wcześniejszym opisem magnetometru można użyć jako kompasu w~celu
wyznaczenia kąta \textit{yaw}.
Jednak w~przypadku przechylenia układu czujników w~pozostałych kątach
płaszczyzna kompasu (magnetometru) zostanie odchylona od płaszczyzny podłoża,
co spowoduje zakłamanie odczytów.
Aby zniwelować ten problem można użyć wyliczonych uprzednio kątów \textit{roll}
oraz \textit{pitch}.
Służy do tego formuła \textit{kompensacji przechylenia} (ang. \textit{tilt
compensation}, którą przedstawia wzory od \ref{eq:tilt_compensation_hx} do
\ref{eq:tilt_compensation_roll}.

\begin{gather}
	h_x = m_x cos\hat{\theta} + 
	      m_y sin\hat{\theta} sin\hat{\phi} +
          m_z sin\hat{\theta} cos\hat{\phi}
\label{eq:tilt_compensation_hx} \\
    h_y = m_y cos\phi - m_z sin\phi
\label{eq:tilt_compensation_hy} \\
    \psi = atan2(-h_y, h_x)
\label{eq:tilt_compensation_roll}
\end{gather}

\subsection{Implementacja programistyczna systemu AHRS}
System AHRS zaimplementowano w~języku C\texttt{++} w~postaci biblioteki
programistycznej.
Aby umożliwić jak najszersze zastosowanie opracowanego kodu zadbano o~jego
wysoką przenaszalność.
Aby, umożliwić działanie programu na urządzaniach wbudowanych \textit{bare
metal} w~programie unikano polegania na wyjątkach oraz dynamicznej alokacji
pamięci.
Większość klas i~funkcji jest parametryzowana przez szablony w~czasie
kompilacji, co pozwala na generowanie szybkiego kodu (kosztem
hipotetycznego zwiększenia rozmiaru plików wykonywalnych).